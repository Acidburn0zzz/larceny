% -*- TeX -*-
%
% $Id: 16:extensions.tex,v 1.1 1992/06/13 08:23:22 lth Exp $

% -*- LaTeX -*-

\documentstyle[10pt]{article}
\newcommand{\reg}[1]{{\sc \%#1}}

\topmargin      -2.0cm
\oddsidemargin   0.0cm
\evensidemargin  0.0cm
\textwidth      6.5in
\textheight     9.5in
\parindent       0.0cm
\parskip         0.4cm

\title{Larceny Note \#16: Extensions to Scheme \\
	{\tenrm (Full of holes\ldots)}}
\author{Lars Thomas Hansen}

\begin{document}
\maketitle

\begin{abstract}
This document attempts to catalogue some of the low-level extensions to Scheme
which we use in order to write as much of the system as possible in Scheme
itself. The fact that these extensions is documented is not a license to use
them in user code; in fact, they may not even be available on the user level.
\end{abstract}

\section{Particular Data Types}

\begin{description}

\item[Bytevector] objects are string-like objects which hold byte quantities.

\item[Bytevector-like] objects are generic string-like objects which hold
byte quantities; in fact, bytevectors, strings, bignums, flonums, and
compnums are all specific kinds of bytevector-like objects.

\item[Vector-like] objects are generic vector objects; other vector types
(proper vectors, symbols, ports, rectnums, and ratnums) are specific kinds of
vector-like objects.

\item[Procedures] look like vectors but smell differently; they have
differently-named operations defined on them. The reason is that procedure
pointers do not look like vector-like pointers.

\end{description}


\section{Typetags}

In objects which have a header (bytevector-like, vector-like, and procedures),
the header holds a type tag for the procedure. This header tag is gettable
and settable by using typetag operations; hence, one type can be efficiently
and violently coerced to another by Scheme code.

For example, one can make a vector object from a vector-like object by 
changing the typetag of the vector-like object.

Changing the typetag of a procedure is allowed but potentially disastrous.

\section{Operations}

The following procedures are implemented on above-mentioned types; some other
extensions to Scheme are also listed here. Most are completely obvious.

\begin{description}

\item {\verb+(bytevector? obj)+}

\item {\verb+(bytevector-ref bv i)+}

\item {\verb+(bytevector-set! bv i v)+}

\item {\verb+(bytevector-length bv)+}

\item {\verb+(bytevector-like? obj)+}

\item {\verb+(bytevector-like-ref bvl i)+}

\item {\verb+(bytevector-like-set! bvl i v)+}

\item {\verb+(bytevector-like-length bvl)+}

\item {\verb+(procedure? obj)+}

\item {\verb+(procedure-ref proc i)+}

\item {\verb+(procedure-set! proc i v)+}

This is an extremely dangerous operation to use if care is not taken, as
changing a procedure can upset the system in very peculiar ways. In particular,
changing a procedure which lives in the static area is very bad, as is changing
a procedure's code vector if the procedure can be returned through in some
live continuation.

\item {\verb+(procedure-length proc)+}

\item {\verb+(vector-like? obj)+}

\item {\verb+(vector-like-ref vl i)+}

\item {\verb+(vector-like-set! vl i v)+}

\item {\verb+(vector-like-length vl)+}

\item {\verb+(typetag obj)+}

The object must be a procedure, a vector-like structure of some sort, or a
bytevector-like structure of some sort, or this operation signals an error.

\item {\verb+(typetag-set! obj t)+}

The object must be as for {\tt typetag}; the tag ``t'' must be a small
positive fixnum in the range 0\ldots 7. The typetag of the object is changed
to the given tag; the old tag is lost. If the object is know to other
code which depends on its former status (i.e., typetag), then that other
code is likely to crash (with an error due to a type violation) later.

\item {\verb+(fixnum? obj)+}

\item {\verb+(bignum? obj)+}

\item {\verb+(ratnum? obj)+}

\item {\verb+(flonum? obj)+}

\item {\verb+(compnum? obj)+}

\item {\verb+(rectnum? obj)+}

\item {\verb+(port? obj)+}

\end{description}

\end{document}
