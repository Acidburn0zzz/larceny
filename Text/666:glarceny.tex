% The Larceny Glossary, or Glarceny
%  Terms and errata

% -*- LaTeX -*-

\documentstyle[10pt]{article}
\newcommand{\reg}[1]{{\sc \%#1}}

\topmargin      -2.0cm
\oddsidemargin   0.0cm
\evensidemargin  0.0cm
\textwidth      6.5in
\textheight     9.5in
\parindent       0.0cm
\parskip         0.4cm

\title{Larceny Note \#666: \\
       The Glarceny\\
       {\tenrm Terms, Errata, and Miscellaneous Concepts possibly related 
         to Larceny}}
\author{R\'{e}my Evard}

\begin{document}
\maketitle

\begin{description}
\item[$\sim$larceny]
The root directory of the larceny development environment.  Each
programmer has their own $\sim$larceny.  Note: $\sim$ is produced in
\LaTeX by saying: {\tt\${\verb+\+}sim\$}.

\item[.c]
The file name extension for C source.

\item[.cfg]
The file name extension for config files, which are processed by
the config utility during the initial build of the larceny executable.

\item[.fasl]
The file name extension for executable Scheme expressions in source form.
The top level read/eval/print loop can load these binary files into the 
system without having
to use the compiler or assembler.  {\tt .fasl} stands for ``fastload''.
See Larceny Note \#6.

\item[.heap]
The file name extension for heap files.
See Larceny Notes \#5 and \#6.

\item[.lap]
The file name extension for Lisp Assembly Programs - MacScheme Assembly
language code in encoded format.  This means that the opcodes have their
numbers filled in (as opposed to {\tt .mal} files).  {\tt .lap} files
are produced by the compiler.
See Larceny Note \#6.

\item[.lop]
The file name extension for Lisp Object Programs - sparc object code that 
may be used to build3 an initial heap heap image.  They are produced from 
{\tt .mal} and {\tt .lap} files by the assembler. 
See Larceny Note \#6.

\item[.mal]
The file name extension for Lisp Assembly Programs - MacScheme Assembly
language in unencoded format.  The opcodes are (human readable) symbols 
that must be processed by the assembler.  {\tt .mal} files are written
by hand.
See Larceny Note \#6.

\item[.map]
The file name extension for map files produced by the heap dumper.
See Larceny Notes \#5 and \#6.

\item[.o]
The file name extension for relocatable object code, produced
by compiling C code or assembling assembly code.

\item[.s]
The file name extension for assembly language.  Currently, all assembly
language is Sparc assembly.
See Larceny Note \#6.

\item[.sch]
The file name extension for scheme files.

\item[.scm]
A file name extension for scheme files.  This extension is considered
archaic and is scorned, or at least not used any longer.

\item[.ss]
The file name extension for scheme files which are Chez specific.  
They live mostly in the $\sim$larceny/Chez directory.

\item[.tex]
The file name extension for \LaTeX files.  The Larceny Notes are written
in \LaTeX.

\item[bignum]
A large exact integer.  ``Large'' means that it is larger than a word.
See Larceny Note \#4.

\item[brownies]
Recommended munching for serious Larceny Note perusal.

\item[bytevector]
A string-like object which holds byte quantities.
See Larceny Note \#16.

\item[bytevector-like]
A generic string-like object which holds byte quantities.
See Larceny Note \#16.

\item[collector]
See garbage collector.

\item[compnum]
Inexact complex numbers.  Both parts are IEEE double precision numbers.
See Larceny Note \#4.

\item[contagion]
Contagion occurs when a numerical operation is to be performed on
numerical representations of two different types.  The appropriate
action is decided upon by a Scheme routine called contagion, which
is called by the generic artithmetic millicode.  See Larceny Note
\#7 for more details.

\item[continuation]
A continuation represents an entire future for a computation.  For
more details, see section 6.9 of R4RS, and Larceny Note \#1.

\item[continuation frame]
?

\item[ephemeral area]
An area of memory used by the garbage collector.  It is used for 
memory allocation.
See Larceny Notes \#3, \#5 and \#8.

\item[exact]
A number is exact if it was written as an exact constant or was derived
from exact numbers using only exact operations. (From R4RS.) 

\item[exception]  An error that happens during execution.  See
Larceny Note \#2 for details.

\item[fixnum]
A small exact integer.  ``Small'' means that it can fit in a word
of memory.
See Larceny Note \#4.

\item[flonum]
Inexact real numbers.  They conform to IEEE double precision format.
See Larceny Note \#4.

\item[garbage collection]  
Automatic memory reclamation.  A win.
See Larceny Notes \#3, \#5 and \#8.

\item[generic arithmetic]
Scheme supports several flavors of numerical representation,
ranging from exact integers to inexact complex numbers.
Generic arithmetic allows numerical operations to be performed
between numbers of any type (or nearly).  Thus, it is possible
to say {\tt (+ 1 2/3 4.5 6+7i)} and get something like
{\tt 12.166666666666668+7.0i} (instead of {\tt core dumped}).
For more details, see Larceny Notes \#4 and \#7, in particular
the comments about contagion.  (Until computers came along, most
arithmetic was generic anyway.)

\item[globals table]
This is where the rooted (and almost all nonrooted) objects are kept.
It is used by the garbage collector.  For more hazy details, see
garbage collection.

\item[hardware register]
?

\item[heap] 
The heap is a way of referring to the ephemeral area, the tenured area,
and possibly the stack area.  The main advantage of the word ``heap'' 
is that it's short and sounds computer-sciency.

\item[heap dumper]
A program which gathers up {\tt .lop} files and collects them into a 
single relocatable, loadable heap, which is then written to a 
{\tt .heap} file.  The heap-dumper is called by {\tt make-heap}
in the development environment.  See Larceny Note \#5.

\item[heap image]
A file containing raw code.  It is loaded at run time, and
intially created by the heap dumper.
See Larceny Notes \#5, \#6, \#8 and \#18.

\item[in-line]
To put the code for a procedure directly into the generated code, rather
than having the code call that procedure.  In general, the more the
code is in-lined, the faster and larger it will be (ignoring cache
behavior).

\item[Index]
A list of files that are part of the Larceny project.  An entry for
a file briefly describes the file.  Each directory has its own Index.
  The Index in $\sim$larceny contains a list of the directories. 

\item[inexact]
A number is inexact if it was written as an inexact constant if it was
derived using inexact ingredients, or was dervied using inexact operations.
(From R4RS.)  Inexactness comes about because of the limited capacity
of real-world machinery to represent infinite numbers.

\item[Larceny]
(1) The felonious taking and removing of another's 
personal property with the intent of permanently depriving the owner; theft.
(2) A lean and mean Scheme system developed at the University of Oregon.

\item[Larceny Notes]
A collection of writings concerning the implementation of Larceny.
A note either covers a single topic, or is an overview of some
part of the system.  They typically are created when the need for
the note becomes apparent, thus they are not in a logical order.  
As of this writing, they are created using \LaTeX, and live in the
directory $\sim$larceny/Text. For more information, as well as
a relatively current list of notes, see Larceny Note \#0.

\item[local debugger]
?

\item[MacScheme]  
A Scheme implementation for the Macintosh by Lightship Software.  
Much of the code for the runtime libraries was written for MacScheme, 
and is copyright Lightship Software, Incorporated.

\item[MacScheme Assembly Language]
Files produced by the compiler are in MacScheme Assembly Language.
See Larceny Note \#5.

\item[millicode]
The parts of the system written in assembly language, although this 
definition will make Lars grimace.  The reason is that some procedures
that are thought of as ``millicode procedures'' are in fact written in
C and then called via assembly stubs.  Also, there are some parts of
the system, such as the start-up code, which are written in assembly
but are not considered to be millicode.  Thus, we are left with this
intuition-based definition: stuff that's not done in-line when the 
assembler generates object code.

\item[millicode support vector]
?

\item[millicode table]

\item[mutator]
An ``application'' using the garbage collector.
See Larceny Notes \#3 and \#8.

\item[non-tail recursion]  
See non-tail recursion, then finish reading this.

\item[overflow]
?

\item[P1178]
? (IEEE standard)

\item[primop]
A procedure that the compiler knows will be in-lined.  Conceptually, 
a ``known'' procedure.  The list of primops varies over time (due to
development) and with optimization level.

\item[procedures]
Procedures ``look like vectors but smell differently'', since procedure
pointers don't look like vector-like pointers.  They have 
differently-named operations.  
See Larceny Note \#16.

\item[R4RS]
The Revised$^4$ Report on the Algorithmic Language Scheme.  This is
the current Scheme Implementor's and Programmer's bible.  Don't leave
home without it.

\item[ratnum]
Exact rational numbers (fractions).  The numerator and the denominator
may be fixnums or bignums.  The denominator must be larger than 1.
See Larceny Note \#4.

\item[RCS]
Revision Control System.  A system for software version control, by 
Walter F. Tichy.

\item[rectnum]
Exact complex numbers.  The real and the imaginary parts are both exact
reals. The imaginary part must be non-zero.
See Larceny Note \#4.

\item[recursion]  
See recursion.

\item[register]
?  (Something about rootable registers.)

\item[root]
A location from which the garbage collector will start a traversal.
There's a fairly small number of roots.

\item[rootable]
A register that will be used as a root during garbage collection.

\item[Scheme 313]
A historical name for some parts of Larceny.  There should be
a Larceny folklore Note about this.

\item[silly recursion jokes]
See the Glarceny.

\item[software register]
?

\item[stack area]
An area of memory used by the garbage collector.  An area of memory
that is managed by the mutator. In Larceny, this area holds the stack 
cache.
See Larceny Notes \#3 and \#8.

\item[stack cache]
Stack frames are kept in the stack cache, which lives in the stack area.
See Larceny Note \#20.

\item[stack cache overflow]
?

\item[stack cache underflow]
?

\item[stack frame]
?

\item[static area]
An area of memory used by the garbage collector.  It is used to store
objects which will never be collected.
See Larceny Notes \#3 and \#8.

\item[tags]
All objects in Larceny are tagged, both for garbage collection purposes
and for type checking.  Tags are documented in Larceny Note \#18.

\item[tail recursion]  
See tail recursion.

\item[tenured area]
An area of memory used by the garbage collector.  It is used to keep
data structures which are assumed to have an extended life.
See Larceny Notes \#3, \#5 and \#8.

\item[thunk]
A procedure which takes no parameters.

\item[transaction list]
The list of pointers to data structures in the tenured area which contains
pointers into the ephemeral area.  It is used by the garbage collector 
during ephemeral collections.
See Larceny Notes \#3, \#5 and \#8.

\item[types]
:-)

\item[typetags]
In objects which have a header, the header hold a type tag.  The tag
can be manipulated.  Tags are defined in $\sim$larceny/Lib/xlib.sch.
See Larceny Note \#16 for more info about manipulating the tag.

\item[twobit]
The compiler.  

\item[underflow]
?

\item[vector-like]
A generic vector object.  Other vector types (like proper vectors, symbols,
etc.) are specific kinds of vector-like objects.
See Larceny Note \#16.

\end{description}
\end{document}
