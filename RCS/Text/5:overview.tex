% -*- TeX -*-
% $id$

% -*- LaTeX -*-

\documentstyle[10pt]{article}
\newcommand{\reg}[1]{{\sc \%#1}}

\topmargin      -2.0cm
\oddsidemargin   0.0cm
\evensidemargin  0.0cm
\textwidth      6.5in
\textheight     9.5in
\parindent       0.0cm
\parskip         0.4cm

\title{Larceny Note \#5: \\
       An Overview of the System \\
       {\tenrm (Draft. Comments welcome.)}}
\author{Lars Thomas Hansen}

\begin{document}
\maketitle

\begin{abstract}
This document has two main parts: Part 1 (sections 1-4) delves into some
gory detail about how to use the Chez Scheme hosted compiler, how to build a
heap, how to load the heap, what debugging facilities are available, and so
on. Part 2 (section 5) describes the implementation in general terms and
attempts to provide some background information.

\end{abstract}

\section{Introduction}

The general structure of running a Scheme program under Larceny is this: the
compiler generates MacScheme assembly language, which the assembler
then converts to files of Sparc object code in a relocatable format.
Several such files are combined into a relocatable heap image, which
is loaded by the Larceny executable at startup time. Sections
2 and 3 explain this process in more detail. Eventually the
compiler and assembler will run as part of the Larceny system, and the
Sparc object code modules will be loaded directly into a running
Scheme system; this is described in section 4.

%%

\section{From Scheme to a Heap}
\label{compiling}

The compiler converts each top-level construct in the source program
to a thunk before compilation proper; for example, the two expressions

\begin{minipage}{\linewidth}
\begin{verbatim}
    (define (foo x)
      ...)

    (foo 33)
\end{verbatim}
\end{minipage}

\noindent become

\begin{minipage}{\linewidth}
\begin{verbatim}
    (lambda ()
      (set! foo (lambda (x)
                  ...)))

    (lambda ()
      (foo 33))
\end{verbatim}
\end{minipage}

\noindent The advantage of this conversion is that it becomes a simple and
uniform task to perform heap initialization, as described below.

Each thunk, when compiled, becomes a list of MacScheme assembly language
instructions where the mnemonic of the instruction is a small positive
fixnum. These mnemonics are defined in the file ``Compiler/pass4.imp.sch''.
Files consisting of such lists of compiled thunks have the file 
extension {\tt .lap}, for ``Lisp Assembly Program''.

The assembler takes such a list of MacScheme instructions as input and
generates as output a {\em segment}: a segment is a pair, the car of
which is a byte vector\footnote{In implementations without byte
vectors, like Chez Scheme, byte vectors are represented using ordinary
vectors.} with the relocatable Sparc object code, and the cdr of which
is the constant vector for that piece of code.  Files consisting of
segments have the file extension {\tt .lop}, for ``Lisp Object
Program''. Segments are described in detail in Larceny Note \#6.

In addition, there is another file format for MacScheme assembly
programs.  Some MacScheme assembly code must be written by hand to get
at the correct low-level primitives. Such code is kept in files with a
{\tt .mal} extension, for ``MacScheme Assembly Language''. The format
of these files is the same as the format of the {\tt .lap} files,
except that when read by the assembler driver, each source expression
in a {\tt .mal} file must be evaluated before it can be assembled: the
opcode mnemonics are unevaluated in these files.  As a typical
example, consider

\begin{minipage}{\linewidth}
\begin{verbatim}
    `((,$lambda ((,$.proc)
                 (,$args= 2)
                 (,$reg 1))
                0
                (#t))
      (,$setglbl foo)
      (,$return))
\end{verbatim}
\end{minipage}

\noindent This expression, when {\em evaluated}, will be a valid input to the
assembler.

The heap dumper takes a number of {\tt .lop} files and combines them into a
relocatable, loadable heap. The heap image consists of a number of root
pointers (which are loaded into the roots table in the running system)
followed by the relocatable heap proper. One of the roots is set to point to
the heap initialization thunk, which is created by the heap dumper. This
thunk has in its constant vector a pointer to a list of all the thunks
loaded into the heap (except itself), and the initialization thunk will,
when run, cdr down this list, calling each thunk, so that the heap can be
initialized properly. The thunks are called in the order in which they were
loaded. When the list has been exhausted, the global (programmer-defined)
procedure ``go'' is called with the list of used symbols as its only
argument; the ``go'' procedure must initialize the symbol table and then it
can do what it wishes. Typically, it calls initializion routines for a
number of system packages and then the read-eval-print loop, or some other
top-level code.

Heaps are stored in files with a {\tt .heap} extension, and heap files
are described in detail in Larceny Note \#6.

The heap dumper also makes a map file (occasionally useful for
debugging) which contains the names of all the global symbols together
with an identifying number for each; the heap dumper places this
number in the cdr of the actual value cell for the global variable.
Hence, when some unknown global is encountered, its identifying number
can be found, and its name can be discovered from the map file.

A more general map file would be nice, but it would not be very useful
after the first garbage collection.

To make a heap, call on the ``make-heap'' procedure in the hosted
environment (defined in ``Lib/makefile.sch''). It takes an argument which is
the filename for the heap to be made, and in addition a number of optional
arguments which are switches to various passes of the compiler, assembler,
and heap dumper, or additional file names to link into the heap.

The switches are these:

\begin{description}
\item[listify] This is a switch to the assembler which tells it to list
the MacScheme code as it is assembled. This has limited utility when things
get big. It is better to use the procedure ``readify-file'' to get the same
effect (see below).

\item[global-refs] This is a switch to the assembler which tells it to
insert checks on global variables when they are loaded into registers; if
a global variable has the value \verb+#!unspecified+, then an exception
about an undefined global variable is raised.

\item[global-symbols] This is a switch to the heap dumper which, when given,
makes the heap dumper initialize the cdr of a global cell with a pointer
to the symbol structure which has the same printname as the global variable.
This can be very useful in debugging. Normally the cdr of a global cell is
initialized to the ordinal number of the cell.

\end{description}

A brief description of what happens behind the scenes is appropriate at this
point. ``Make-heap'' is a procedure (defined in ``Lib/makefile.sch'') which
creates a dependency list based on the arguments and an internal variable
which holds the list of files which go into a default heap image. In
addition, the switches, if any, trigger the setting or resetting of various
global variables in different modules. The dependency list is passed to the
``make'' procedure, a simple UNIX-like ``make'' facility, defined and
documented in ``Compiler/make.sch''. This utility traverses the dependency
tree and recompiles targets which are out of date with respect to their
sources.  Custom versions of ``make-heap'' for other library configurations
should be straightforward. Anyway, to compile a ``.sch'' file to a ``.lap''
file, the procedure ``compile313'' is used, and to assemble a ``.mal'' or
``.lap'' file to a ``.lop'' file, the procedure ``assemble313'' is used.
Both procedures are described below. Finally, the procedure ``dump-heap'' is
called to dump a heap containing the ``.lop'' files in its argument list;
this procedure is defined in ``Compiler/dumpheap.sch''.

\section{The Hosted Compiler.}

Table \ref{files} presents all files which are part of the compiler,
assembler, and heap dumper.

\begin{table}[hbt]
\begin{center}
\begin{tabular}{|l|l|} \hline
Sparc/asm.sparc.sch       & Sparc assembler \\
Sparc/disasm.sparc.sch    & Sparc disassembler \\
Sparc/gen-primops.sch     & Code emitters for ``op'' instructions \\
Sparc/gen-msi.sch         & Code emitters for MacScheme instructions \\
Compiler/assembler.sch    & Assembler top level (generic) \\
Compiler/build.sch        & Compiler system bootstrap \\
Compiler/compile313.sch   & Drivers for compiler, assembler, disassembler \\
Compiler/dumpheap.sch     & Heap dumper \\
Compiler/pass1.sch        & Macro expansion \\
Compiler/pass1.aux.sch    & \\
Compiler/pass1.imp.sch    & Primop tables \\
Compiler/pass2.sch        & Closure analysis \\
Compiler/pass4p1.sch      & Code generation \\
Compiler/pass4p2.sch      & \\
Compiler/pass4.aux.sch    & \\
Compiler/pass4.imp.sch    & MacScheme instruction mnemonics \\
Compiler/printlap.sch     & Utility \\
Compiler/sets.sch         & Set operations for compiler \\
Compiler/utils.sch        & More utilities \\
Compiler/make.sch         & ``Make'' utility \\
Lib/makefile.sch          & Makefile for Larceny heaps \\
Chez/*                    & Compatibility package \\ \hline
\end{tabular}
\end{center}
\caption{Files in the hosted system}
\label{files}
\end{table}

The batch compiler hosted under Chez Scheme can be built by loading
the file ``Compiler/build.sch''. You should be in the ``Compiler''
subdirectory when starting up Scheme, as all paths in the files are
relative to that directory. The build script will in turn load the
compiler, Sparc assembler, heap dumper, compatibility package, make
utility, and other utilities. In general, the following procedures are
useful:

\begin{description}
\item {\verb+(compile313 <filename>)+}

Compile the source file with extension ``.scm'' or ``.sch'' (Scheme source
files) to a file with the same path and base name but with extension
``.lap'' (MacScheme assemly language code).

\item {\verb+(assemble313 <filename>)+}

Assemble a file with extension ``.lap'' or ``.mal'' (MacScheme assembly
language files) to one with extension ``.lop'' (compiled and assembled
Sparc code in segment form).

\item {\verb+(disassemble313 <inputfile> [ <outputfile> ] )+}

Disassemble a file on segment form (``.lop'' type); the code vectors are
disassembled to Sparc assembly language. Constant vectors are not shown,
but code vectors in the constant vector are disassembled. The code vectors
are labelled somewhat coherently.

\item {\verb+(dump-heap <outputfile> <inputfile> ...)+}

Collect all the ``.lop'' input files and create a heap as the output file.
All thunks are gathered in a list and will be called in turn (in load order)
when the heap is loaded into Larceny; finally, there must in one of these
files be a procedure ``go'' of one argument which will be called after all
the thunks have been called. The argument passed to ``go'' will be a list
of symbols in the load image; ``go'' may do what it wants with them.
``Go'' should never return.

\item {\verb+(readify-file <inputfile> [ <outputfile> ] )+}

List a ``.lap'' file by converting the mnemonic codes for MacScheme 
instructions to human-readable form.

\item {\verb+(make-heap <filename> [ <argument> ... ] )+}

Run the make utility to make a heap with the given name. The arguments to
this command are described above.

\item {\verb+(make <target> <dependencies>)+}

You can use the make utility directly if you create a proper list of
dependencies. The gory details are in the comments at the head of the file
``Compiler/make.sch''.

\end{description}

\section{Running Larceny}

Once you've managed to make the executable (say ``make'' at the UNIX prompt
when you're at the root of the build tree), you can invoke Larceny with
the command ``larceny'' and appropriate options. 

\subsection{Command line options}

The following options are currently (sort of) recognized:

\begin{description}
\item {\tt -t nnnn}

 Tenured heap size in bytes (decimal); defaults to 1 MB.
\item {\tt -e nnnn}

 Ephemeral heap size in bytes (decimal); defaults to 1 MB.
\item {\tt -s nnnn}

 Stack cache size in bytes (decimal); defaults to 64K.
\item {\tt -I nnnn} 

Initial value of the interval timer (decimal); defaults
to \verb+#xFFFF+.

\item {\tt -h filename}

 Heap file. This option must be supplied (and so it
is not really an option).

\item {\tt -b n}

 Stop at (dynamic) breakpoint number n.

\item {\tt -B}

 Stop at all breakpoints.
\end{description}

At the present time the switches {\tt -t}, {\tt -e}, and {\tt -s} are 
silently ignored. The {\tt -b} and {\tt -B} switches will throw you
into the ``local'' debugger,\footnote{Someone should do some research to find
out how we come up with nonintuitive names like that.} a low-level
facility of limited utility.

\subsection{The Local Debugger}

There is an online help facility which explains the available commands 
superficially. Basically, one can do register and procedure dumps, and one
can inspect arbitrary pieces of system memory. Be warned that with the
exception of registers, tags are not stripped from addresses, so if 
you are not careful, a bus error will result.

One of the more useful debugging techniques at this level is to dump the
current code vector (with the 'c' command) to the file ``/tmp/larceny'' and
then disassemble it using the ``disassemble313'' procedure in the
development environment.

\subsection{Some implementation details}

The executable first parses its command line. It then initializes the memory
subsystem and loads the heap image into memory. Finally it calls the
procedure schemestart(), which is the entry point to Scheme procedures in
the heap.

Schemestart() grabs the entry point out of the globals table (which was set
up when the heap was loaded) and jumps to it. For an uninitialized heap,
this entry point is the initialization thunk. If the thunk returns,
Larceny will exit with a status of 0.

\section{The Run-time System}

The run-time system consists of the following major parts:

\begin{itemize}
\item Memory management (storage allocation and reclamation)
\item Generic arithmetic support
\item Library code
\end{itemize}

This seems awfully simple, and indeed it is deceptive: much of the
work in the system has not been to implement these parts, but rather
to design calling conventions and register usage strategies, and to
write the compiler and assembler. Nevertheless, only the above parts
will be described here.

\subsection{Memory Management}

One of the major advantages of higher-level languages like Scheme is
the availability of automatic storage allocation and reclamation. All
responsibility for maintaining storage is taken away from the programmer
and given to the run-time system. The cost of this is that the run-time
system must provide a garbage collector to reclaim storage no longer in
use. The presence of a garbage collector, in turn, imposes several
constraints on the design of the rest of the run-time system, as well as
on calling conventions, stack layout, and other performance-affecting
parts of the system.

First, all data must be tagged so that the garbage collector can
figure out what kind of data it is dealing with and how to collect it
properly. In Larceny, each word is tagged with a 3-bit tag which
uniquely identifies it (the exception being data in bytevectors; see
the next section).  Tags are kept in the low three bits of a word.

Second, registers must be classified as either rootable or
nonrootable.  Rootable registers are used as starting points
(``roots'') for the garbage collector as it traces pointers around
data structures; nonrootable registers are used as temporaries. There
is a need for nonrootable registers because not all bit patterns
represent meaningful Scheme objects, and in order for the collector
not to be confused, we store such bit patterns only in places where
the collector cannot get to them. Similarly, it is a requirement that
all live objects must be reachable by following links from a root.

Third, stack frames must be laid out in such a fashion that the
collector knows what is contained in each word in the frame, and such
that it can find the boundaries between frames.

Fourth, data structures must be laid out carefully so that the collector
can easily traverse them, while at the same time, the structures must allow
efficient accessing.

\subsubsection{Overview of Data Structure Layout}

This material is presented in depth in Larceny Notes \#1 and \#8; consider
the following a brief summary of the highlights.

Objects are represented as 32-bit tagged words. The three low bits are the
tags, although for some data types the high bit of the tag is used as a part
of the object (details follow). These are the tagged objects:

\begin{minipage}{\linewidth}
\begin{verbatim}
  xxxx xxxx  xxxx xxxx  xxxx xxxx  xxxx xx00   fixnum
  xxxx xxxx  xxxx xxxx  xxxx xxxx  xxxx xx10   immediate
  pppp pppp  pppp pppp  pppp pppp  pppp p001   pointer to pair 
  pppp pppp  pppp pppp  pppp pppp  pppp p011   pointer to vector struct
  pppp pppp  pppp pppp  pppp pppp  pppp p101   pointer to bytevector struct
  pppp pppp  pppp pppp  pppp pppp  pppp p111   pointer to procedure struct
\end{verbatim}
\end{minipage}

Immediates are used for odd data around the system. The following are
the immediate formats.

\begin{minipage}{\linewidth}
\begin{verbatim}
  0000 0000  0000 0000  0000 0000  0000 0010   #f
  0000 0000  0000 0000  0000 0000  0000 0110   #t
  0000 0000  0000 0000  0000 0000  0000 1010   empty list
  xxxx xxxx  xxxx xxxx  xxxx xxxx  0001 0110   miscellaneous
  0000 0000  cccc cccc  0000 0000  0010 0110   character
  0sss ssss  ssss ssss  ssss ssss  100x xx10   reserved header
  0sss ssss  ssss ssss  ssss ssss  101x xx10   vector-like header
  0sss ssss  ssss ssss  ssss ssss  110x xx10   bytevector-like header
  0sss ssss  ssss ssss  ssss ssss  1111 1110   procedure header
\end{verbatim}
\end{minipage}

The "s" bits must contain the size of the data structure in bytes, not
including the header word, and not including padding (see below). The
collector will correctly round "bytevector" and "reserved" lengths up
to a word boundary. Hence, for these structures, the "s" bits give the
correct length of the datum, like the length of a string. For vectors
and procedures the s field must be evenly divisible by 4 (low two bits
are 0).

Since all pointers are doubleword-aligned, some vectors, bytevectors,
and procedures will have to be padded out to an even number of words.
The mutator must allocate the extra word but may choose to leave it
uninitialized; the collector will take the size of the vector into
account and skip the padding. If the word is initialized, it may or
may not be copied along with the structure during a collection.

Tagged pointers point to the following structure types:

\begin{itemize}
\item
Pairs: A pair has two words, the car (low word) and the cdr (high word).
The pair pointer points to the car of the pair.

\item
A vector-like structure has the header word in the low position, followed
by tagged pointers in all locations of the vector. The vector pointer
points to the header word.

\item
A bytevector-like structure has the header word in the low position,
followed by untagged bytes in all locations of the vector. The bytevector
pointer points to the header word.

\item
A procedure structure has the header word in the low position, followed by
tagged pointers in all locations of the procedure structure.  With the
exception of the tag, a procedure looks just like a vector-like structure.
The procedure pointer points to the header word.
\end{itemize}

\subsubsection{An Efficient Garbage Collector}

It has long been clear that the memory allocation behavior of
higher-level languages like Scheme is characterized by the allocation
of varying amounts of short-lived memory and, over time, rather
larger amounts of long-lived memory. Most reasonable garbage
collectors use a copying approach (know as stop-and-copy): objects are
allocated in an area of memory known as ``oldspace'', and when that
area fills up, the live objects are traced from a set of roots and
copied into a ``newspace'', with the necessary pointer adjustments.
Then, ``newspace'' becomes the new ``oldspace'' and ``oldspace''
becomes the new ``newspace'', and execution continues. There are two
main advantages of this method of garbage collection: only live
objects are referenced, and compaction of the heap is automatic.
Another popular technique known as mark-and-sweep both references all
objects (live and dead) and does not compact automatically.
Larceny uses a stop-and-copy approach.

However, due to the way the collector works, it is advantageous to
collect only among those objects which are likely to be ``dead''
(since dead objects are not referenced) and to avoid objects we know
to be live. Taking the memory allocation behavior of Scheme programs
into account, we divide the heap into two generations: an
``ephemeral'' generation, and a ``tenured'' generation, each of which
has an oldspace and a newspace. All objects are allocated in the
ephemeral generation, and when a garbage collection happens, live
objects are copied from the ephemeral oldspace to the ephemeral
newspace; this is know as an ephemeral collection. If, after such a
collection, the ephemeral newspace is fuller than some limit, the {\em
next} collection will copy all the live objects from the ephemeral
oldspace into the tenured newspace; this is known as a tenuring
collection. After that, the system goes back to allocating and
collecting in the ephemeral area again, until it fills up anew, and a
tenuring collection has to be performed.  Eventually, the tenured
newspace will fill up, and a collection of tenured space will have to
be performed; this is know as a full collection.\footnote{Also ``moby''
collection.} When the tenured newspace is full after a collection,
the system panics. This can be fixed, but it is not worth 
worrying about at the moment.

In general, the full collection should not necessarily be of the copying
kind, as it means having a large number of megabytes of unused space
floating around at all times (for the tenured newspace). This, too, can be
fixed but is of little importance for the current system.

In Larceny there are two more heap areas, known respectively as the
``static'' and ``stack'' areas. The static area holds objects which
will never be collected, like system code. The stack area holds the
stack cache, about which more will be said later.

\subsubsection{The Problem of Side Effects}

The generational collection scheme creates a problem, however, in
languages in which side effects on data structures are allowed.
Consider the case where a field in a data structure in the tenured
space is made to point to an object which has been newly allocated in
the ephemeral space. If this pointer is the only reference to the
object in the ephemeral space, it must be found, otherwise the object
will not be copied. If it is not the only link, then the object may be
copied, yet the pointer will not be changed to point to the new
location of the object. In short: all pointers from the tenured
generation into the ephemeral generation must be found and updated
when an ephemeral collection occurs. While there are several ways to
do this, the one used by Larceny involves the maintaining of a
{\em transaction area} which contains pointers to all objects in the
tenured space which in turn contain pointers into the ephemeral space.
Whenever a side effect (vector-set!, set-car!, set-cdr!, and system
procedures like cell-set!) is performed, and the object being changed
is in the tenured space, a transaction is recorded in the transaction
area. Then, when an ephemeral collection is performed, the transaction
area is traversed and all pointers into the ephemeral area can be followed
and updated.

\subsubsection{Efficient First-class Continuations}

One of the unique features of Scheme is the access it grants the
programmer to the continuation structures. However, if the programmer
is going to be interested in using this feature (and it is an
interesting and useful concept), it must be efficiently implemented.
A number of strategies for efficient implementation of the system
procedure ``call-with-current-continuation'' (abbreviated ``call/cc'')
have been proposed; the one used in Larceny is known as the incremental
stack/heap approach and works as follows.

There is a stack cache of limited size; the default is 64KB.
Continuation frames are created on this stack until it overflows or
until the continuation is captured with ``call/cc''. When either of
these happen, the stack cache is broken up into individual frames
which are moved to the heap (and where the frames form a linked list,
linked by the dynamic links of the frames). When the stack cache
underflows or a continuation is reinstated, a single frame is fetched
into the cache again before execution continues (the reinstating of a
continuation causes the stack cache to be cleared before the fetching
of the frame). Thus, if ``call/cc'' is called again, most of the
continuation may already be in the heap,\footnote{The common case
appears to be that continuations share much of their structures.} and
only a few frames have to be flushed.

The overhead in this strategy is the checking for stack cache overflow
on frame creation. There need not be any overhead to detect underflow;
one places a dummy frame on the bottom of the stack cache, which, when
returned into, jumps to the underflow handler and reinstates a frame
before jumping to the correct return address.

\subsection{Generic Arithmetic Support}

One of the goals in implementing Larceny was to get a testbed for research
in fast generic arithmetic. Consequently, the support for generic arithmetic
is extensive and comprises a large part of the run-time system; the basic
arithmetic operations weigh in at about 4000 lines of code, half of it in
Scheme and half of it in Sparc assembly language.

\subsection{Representation}

There are six representations for numbers.

\begin{itemize} 
\item Fixnums fit in a tagged word and
can represent small exact integers.
\item Bignums are represented as bytevectors
and represent exact integers which will not fit in a fixnum.
\item Ratnums
are represented by a vector (numerator and denominator, both are in turn
numbers) and represent exact rationals with a denominator greater than 1.
\item Rectnums are also represented as vectors and represent exact complex 
numbers
with a non-zero imaginary part. 
\item Flonums are represented as bytevectors
and represent inexact real numbers.
\item  Compnums are represented as bytevectors
and represent inexact complex numbers, possibly with a 0 imaginary part.
\end{itemize}

\subsection{Implementation}

In the absence of type information at compile time, all fixnum arithmetic is
done in line, and the rest is pushed into millicode.  Larceny Note \#7 goes
into detail; the following is just an overview.

If the integer operation fails or is too heavyweight, the generic arithmetic
operation falls into the millicode. The millicode does a peculiar kind of
type dispatch to efficiently figure out whether it can do same-representation
arithmetic on fixnums, flonums, and compnums. If this dispatch fails,
millicode sets up a continuation for the original Scheme code and falls
into Scheme to complete the operation. Operations on mixed-representation
numbers as well as operations on bignums, ratnums, and rectnums are
all done in Scheme.

In the case of mixed-representation arithmetic and comparison, the operands
are promoted to a common type and the operation is retried. The promotion
is straightforward in most cases; the exception is when we have a comparison
operator and one bignum and one flonum operand. In this case, in order to
have transitive operations, we have to coerce one way or another depending
on the value of the flonum. If the flonum represents an integer, it is 
``promoted'' to a bignum for the sake of the comparison and the operation
is retried with bignums. Otherwise, the bignum is converted to a flonum
and the operation is retried with flonums.

% \subsection{The Run-time Library}

\end{document}
