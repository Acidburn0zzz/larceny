% -*- TeX -*-
% Canonical List of Larceny Notes
% $Id$

% -*- LaTeX -*-

\documentstyle[10pt]{article}
\newcommand{\reg}[1]{{\sc \%#1}}

\topmargin      -2.0cm
\oddsidemargin   0.0cm
\evensidemargin  0.0cm
\textwidth      6.5in
\textheight     9.5in
\parindent       0.0cm
\parskip         0.4cm

\begin{document}

\bigskip
\centerline{\Large List of Larceny Notes}

\bigskip

Please be aware of the nonintuitive numbering of these notes. Notes
are numbered as they are conceived of and written, not as they would
be ordered in a logical exposition of the material.

For a general introduction, read notes \#11 and \#5.

If you find bugs in these notes, correct the on-line copies and make a note
in the paper copy so that a new paper copy can be printed when the number
of changes justifies killing another tree.

\begin{description}

\item {\tt conventions.tex}

Larceny Note \#1: Register Usage, Continuation Layout, and Calling Conventions.

\item {\tt exceptions.tex}

Larceny Note \#2: Exception Handling.

\item {\tt memory.tex}

Larceny Note \#3: Memory Management Millicode.

\item {\tt numrepr.tex}

Larceny Note \#4: Number Representations.

\item {\tt overview.tex}

Larceny Note \#5: An overview of how to use the compiler in the hosted
environment, how to invoke Larceny, and other odds and ends. Also an
overview of the basic workings of the runtime system.

\item {\tt fileformats.tex}

Larceny Note \#6: File formats.

\item {\tt arithmetic.tex}

Larceny Note \#7: Generic Arithmetic Implementation.

\item {\tt gc.tex}

Larceny Note \#8: The Garbage Collector and the Memory Manager.

\item {\tt assembler.tex}

Larceny Note \#9: The Hitchhiker's Guide to the Assembler.

\item {\tt schemeio.tex}

Larceny Note \#10: The Larceny I/O System.

\item {\tt intro.tex}

Larceny Note \#11: Introduction to Larceny (ideas \& philosophy).

\end{description}

\end{document}
