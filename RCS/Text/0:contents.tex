% -*- TeX -*-
% Canonical List of Larceny Notes
% $Id: 0:contents.tex,v 1.2 1992/05/08 18:39:32 lth Exp lth $

% -*- LaTeX -*-

\documentstyle[10pt]{article}
\newcommand{\reg}[1]{{\sc \%#1}}

\topmargin      -2.0cm
\oddsidemargin   0.0cm
\evensidemargin  0.0cm
\textwidth      6.5in
\textheight     9.5in
\parindent       0.0cm
\parskip         0.4cm

\title{Larceny Note \#0: A Meta-Note}
\author{Lars Thomas Hansen}

\begin{document}
\maketitle

\section{Introduction}

``Larceny Notes'' is a collection of writings on the implementation of
Larceny, a run-time system for Scheme on the SPARC architecture. Each
Larceny Note typically covers a single subject and was usually written (or
at least its contents was outlined even if the note was not actually
written) at some point when a feature was nailed down or when massive
confusion was apparent and documentation of the implementation became
necessary. Hence, the notes have no logical organization (except internally
to each one); a reading guide is presented below. A few notes describe
features which have not been implemented, but which should be at some point.

The collection of Larceny Notes occasionally gets updated collectively to
reflect the true state of the system at a given point. The title page
for the Larceny Notes reflects the version number of Larceny at the time
of such an update. The notes then have a tendency to diverge with respect
to how they reflect the true implementation and status of the system, and
new notes are added describing new  or previously undocumented features.

Specifically, ``Larceny Notes'' includes descriptions of the run-time system
and the assembler (MacScheme assembly language to Sparc object code
translator), including file formats accepted and produced by these
subsystems.  In particular, ``Larceny Notes'' does not include a description
of the compiler, as the compiler technology is not relevant to the run-time
system.  Neither is there much of a description of the parts of the run-time
which are written in Scheme, although bits and pieces of the really low-level
stuff is described in various places.

Please be aware of the nonintuitive numbering of these notes. Notes
are numbered as they are conceived of and written, not as they would
be ordered in a logical exposition of the material.

\section{Reading Guide}

Section 3 lists all the notes in this edition of ``Larceny Notes''. For a
general introduction, read notes \#11 and \#5. For a description of
data representations, read notes \#17, \#1, and \#4. For basic information
about the run-time system, read notes \#1, \#8, and perhaps \#3.

\section{Canonical List of Larceny Notes}

\begin{enumerate}
%1
\item Register usage, continuation and procedure layout, and 
the standard calling conventions.
%2
\item Exception handling: overview of issues, and the current implementation.
%3
\item Memory management millicode.
%4
\item Number representations.
%5
\item An overview of how to use the compiler in the hosted
environment, how to invoke Larceny, and other odds and ends. Also an
overview of the basic workings of the runtime system.
%6
\item File formats.
%7
\item Generic arithmetic and related issues.
%8
\item Garbage collection and memory management.
%9
\item The hitchhiker's guide to the assembler.
%10
\item The Larceny I/O system.
%11
\item Introduction to Larceny (ideas \& philosophy).
%12
\item The MacScheme computational model.
%13
\item The MacScheme instruction set.
%14
\item ``What if\ldots''  --  future enhancements.
%15
\item Nonstandard calling conventions.
%16
\item Extensions to Scheme.
%17
\item Basic Representations.
%18
\item Bootstrapping, Startup, and Magic Variables.
%19
\item A Proposal for a Foreign Function Interface.
%20
\item A Proposal for Fast and Adjustable Stack Caches.
%21
\item The Larceny Development Environment.
%22
\item Internal Naming Conventions.
%23
\item Global Invariants.
%24
\item A Retrospective on the Bootstrapping and Debugging of Larceny.

\end{enumerate}

\end{document}
