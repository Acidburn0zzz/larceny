% -*- TeX -*-
% $Id: 1:conventions.tex,v 1.3 1992/02/23 16:56:50 lth Exp lth $

% -*- LaTeX -*-

\documentstyle[10pt]{article}
\newcommand{\reg}[1]{{\sc \%#1}}

\topmargin      -2.0cm
\oddsidemargin   0.0cm
\evensidemargin  0.0cm
\textwidth      6.5in
\textheight     9.5in
\parindent       0.0cm
\parskip         0.4cm


\title{Larceny Note \#1: \\
       Register Usage, Continuation Layout, and Calling Conventions}
\author{Lars Thomas Hansen}

\begin{document}
\maketitle

\begin{abstract}
This document discusses the mapping of virtual-to-physical registers, the
use of the Scheme stack, the calling conventions for Scheme and millicode,
and the layout of procedures and continuations.
\end{abstract}

\section{Register Usage}

The mapping of virtual to physical registers is defined in the header
file ``registers.s.h'', and is also summarized below. That file
also define symbolic names for registers for use by assembly language
procedures.  The following discussion highlights some of the
trade-offs.

The Sparc version 8 manual states that in order to maintain Sparc ABI
compliance, only the global registers \%g1-\%g4 are available, while
\%g5-\%g7 are reserved for systems use. It is unclear what constitutes
``systems use'', and it appears that if we don't care about Sparc ABI
compliance, we can use all the global registers. Since our register
model already violates the Sparc ABI spec, there is little point in
staying away from the global registers.

The ``call'' instruction uses \%o7 to store the return address; we
cannot in general use the ``call'' instruction for calls in the Scheme calling
conventions, but it is convenient to simulate a ``call'' using
the ``jmpl'' instruction, because calling the millicode becomes
cheaper that way (this is discussed in some detail below).  Therefore,
we should reserve \%o7 for miscellaneous operations involving a return
address.

Likewise, we will want two stacks: one for C and one for Scheme. This
choice lets us tailor the Scheme calling conventions without having to
worry about interrupts and related problems, which reqire a specific
stack layout. The C stack follows the conventions necessary to deal
with asynchronous events. Since the C stack pointer must at all times
be in register \%o6, we cannot use this register for Scheme.

Since the operating system window overflow trap handler uses the stack
pointers of previous windows (i.e. \%i6 after a save), the \%i6
register cannot be used by Scheme, either.

Although \%i7 holds the return address of the previous procedure in the
C calling conventions, we can simply save it somewhere (typically on
the C stack), so \%i7 is available.

Hence we are left with 28 registers available to Scheme: \%g1-\%g7,
\%i0-\%i5, \%i7, \%l0-\%l7, and \%o0-\%o5. In addition, \%g0 is hardwired
to the value 0. We have the following virtual and work register classes to
map onto those registers (their function is elaborated on later):

\vfill
\begin{description}

\item {\bf REG0 ... REGn, for some n}

These are the general-purpose virtual registers. A subset must be
mapped to physical registers, and the rest will be in memory.
Rootable.

\item {\bf RESULT}

The register in which we return a value and pass argument counts. Also used
to pass the first argument to millicode procedures. Rootable.

\item {\bf ARGREG2}

Second argument to a millicode procedure. Rootable.

\item {\bf ARGREG3}

Third argument to a millicode procedure. Rootable.

\item {\bf STKP}

Scheme stack pointer. Nonrootable.

\item {\bf E\_TOP, E\_LIMIT, and others (perhaps STK\_LIMIT)}

Heap pointers. Nonrootable.

\item {\bf GLOBALS}

A register pointing to an array of global variables. Nonrootable.

\item {\bf MILLICODE}

A register pointing to a vector of pointers to millicode procedures. 
Nonrootable. This register could be the same register as GLOBALS; in the
current implementation they are different registers.

\item {\bf TIMER}

The countdown timer register. Nonrootable.

\item {\bf TMP0 \ldots TMPm, for some m}

Temporary registers. Nonrootable.
\end{description}

It is convenient for the purposes of code generation to have
even-numbered general-purpose registers mapped to even-numbered
machine registers, as this saves instructions and cycles in some cases
when a "store double" instruction can be used. This also requires that
the Scheme stack pointer always be double-word aligned.

The number of temporary registers needed is unclear; the more the better is
a good rule, but there is clearly a limit. For reasons which will become
clear later, \%g1 should be mapped to a TMP register.\footnote{Actually,
an output register would do, but those are all taken by other stuff, so
the point is still valid.} Also, some two of the
temporaries should be allocated so that they can be used as a register pair,
since this allows certain tweaks.

The current register mapping is shown in table \ref{mappings}.
Note that \%g1 is a temporary, and that \%TMP1/\%TMP2 is a temporary-register
pair.


\begin{table}[hbt]
\begin{center}
\begin{tabular}{|l|l|} \hline
Logical & Physical \\ \hline\hline
   REG0 ... REG7 &  l0 ... l7  \\
   RESULT        &  o0 \\
   ARGREG2       &  o1 \\
   ARGREG3       &  o2 \\
   STKP          &  o3 \\
   E\_TOP        & i0 \\
   E\_LIMIT      & i1 \\
   GLOBALS       &  i7 \\
   MILLICODE     &  i5 \\
   TIMER         &  i4 \\
   TMP0          &  g1 \\
   TMP1          &  o4 \\
   TMP2          &  o5 \\ \hline
\end{tabular}
\end{center}
\caption{Virtual-to-physical register mapping}
\label{mappings}
\end{table}

The symbolic names \%SAVED\_RESULT, \%SAVED\_ARGREG2, \%SAVED\_ARGREG3,
\%SAVED\_STKP, \%SAVED\_TMP0, \%SAVED\_TMP1, and \%SAVED\_TMP2 can be 
used to access their obvious counterparts after a ``save'' instruction has
been executed.

Registers not mapped to hardware registers are mapped to their save locations
in the globals table.

\section{Continuations}

There are two continuation layouts: in the stack and on the heap. Heap
continuations are never manipulated directly by compiled Scheme code;
instead, they are created on stack overflow or on the capture of a
continuation, and restored into the stack on a stack underflow or the
reinstatement of a continuation.

Stack frames have the following layout (memory addresses increasing
going down the figure):

\begin{minipage}{\linewidth}
\begin{verbatim}
%STKP --> pppp pppp pppp pppp pppp pppp pppp pp00   return address slot
          ssss ssss ssss ssss ssss ssss ssss ss00   frame size (bytes)
          pppp pppp pppp pppp pppp pppp pppp p111   procedure (saved REG0)
          xxxx xxxx xxxx xxxx xxxx xxxx xxxx xxxx   argument  (saved REG1)
          ....
          xxxx xxxx xxxx xxxx xxxx xxxx xxxx xxxx   temporary
          ....
\end{verbatim}
\end{minipage}

The frame size is the number of bytes used by the frame, including the
return address and size fields.  A frame should always be an even
number of words, but the size field should indicate the number of
actually used words, i.e. less any padding words. This doubleword
alignment of the stack lets us save some instructions in some cases.
The unused word of the frame should always be the highest addressed word
in the frame, and it may remain uninitialized, if desired: the stack
flushing routine deals with it in a coherent way.

As a special case the procedure word may be all 0's, in which case the
return address will not be adjusted when the stack frame is flushed to the
heap (see note \#8 for details). This allows the creation of frames which
correspond to nonrelocatable code.

When the stack cache is flushed into the heap, each frame is made into
a vector structure with the following layout:

\begin{minipage}{\linewidth}
\begin{verbatim}
      --> 0sss ssss ssss ssss ssss ss00 101t tt10   vector header word
          xxxx xxxx xxxx xxxx xxxx xxxx xxxx xxxx   dynamic link or #f
          xxxx xxxx xxxx xxxx xxxx xxxx xxxx xx00   return offset
          pppp pppp pppp pppp pppp pppp pppp p111   procedure (saved REG0)
          xxxx xxxx xxxx xxxx xxxx xxxx xxxx xxxx   argument  (saved REG1)
          ....
          xxxx xxxx xxxx xxxx xxxx xxxx xxxx xxxx   temporary
          ....
\end{verbatim}
\end{minipage}

The return offset is the offset into the code vector of the saved
procedure to which point the return address pointed before the frame
was flushed to the heap. The code vector is a byte vector; the offset
is from the start of the first data byte of the bytevector, i.e. it is
a valid bytevector index.  Note that all words in the heap
continuation frame must be valid scheme objects; hence, if there is a
chance that a frame is flushed before all temporaries are used, then
the temporaries must have been initialized.

The tag of a continuation frame is an ordinary vector tag.

\section{Procedure layout}

A procedure looks much like a vector (although it has a distinguished
header and is pointed to by a distinguished pointer):

\begin{minipage}{\linewidth}
\begin{verbatim}
     --> 0sss ssss ssss ssss ssss ssss 1111 1110    header
         pppp pppp pppp pppp pppp pppp pppp p101    bytevector ptr (to code)
         vvvv vvvv vvvv vvvv vvvv vvvv vvvv v011    vector ptr (to constants)
         llll llll llll llll llll llll llll llll    static link
         xxxx xxxx xxxx xxxx xxxx xxxx xxxx xxxx    data slot
         ....
\end{verbatim}
\end{minipage}

The data slots contain the contents of variables which were closed
over at run time, and the static link points to another procedure
structure. Top-level procedures do not have static links nor data slots.


\section{Calling conventions}

The discussion of procedure calls separates into several parts:
Procedure entry; setting up for a call; doing a tail call; doing a
non-tail call; and returning from a call. 

One operation which is used quite a bit is the equivalent of the ``lea''
(load effective address) instruction on the mc68000. It calculates the
effective address of some location based on the current PC.  However,
the Sparc has no ``lea'' instruction, nor is the PC available to other
instructions than the ``call'' instruction. On the other hand, since the
``call'' instruction is PC-relative, we can use it to extract the PC and
calculate an effective address from there. So assume we want to say

\begin{verbatim}
        lea     L2, %TMP0               ! compute effective address of L2 into %TMP0
\end{verbatim}

then we really say

\begin{minipage}{\linewidth}
\begin{verbatim}
        call    .+8                     ! "." means "current pc"
        add     %o7, (L2-(.-4)), %TMP0
\end{verbatim}
\end{minipage}

This setup is reasonable as long as \%o7 is not used in compiled Scheme
code, which it isn't. (One of the reasons why \%o7 should not be used
for the return address is because it is needed as a work register for
``lea''.)  The above construct is also fast -- a call completes in 1
cycle, so the entire operation takes only 2 cycles and 2 instructions.
Furthermore, by manipulating the branch address of the call, we can do
a branch and ``lea'' all-in-one.

Special care must be taken when the distance between the ``lea'' and the 
label it addresses is not bounded by some small constant (4K); if it is
the case that the distance can be larger, more instructions are necessary
to calculate the correct address because the offset won't fit in the
immediate of the ``add'' instruction.

\section{Procedure entry}

We have a caller-saves dicipline, so the callee must do very little at
procedure entry time. Typically all that is done is to check the number
of arguments, and even doing so is not mandated by the calling conventions,
but is rather a compiler issue.

In the case of a fixed number of arguments, the check is straightforward
(ARG\_COUNT is the number of expected arguments):

\begin{minipage}{\linewidth}
\begin{verbatim}
L0:
     cmp   %RESULT, ARG_COUNT
     be,a  L1
     nop
     jmpl  %MILLICODE + M_ARG_EXCEPTION
     add   %o7, L0-(.-4))-8, %o7          ; subject to change
L1:
\end{verbatim}
\end{minipage}

The return address given to the exception handler is that of the procedure
entry point, so that the operation can be retried.\footnote{Exception
handling semantics and mechanisms are still in rough shape.} Exception
handling semantics is covered in detail in Note \#2.

Notice the adjustment of "-8" on the effective address calculations.
The adjustment is necessary because a standard return jumps to the
instruction at (return address + 8).  Since there is no sense in using
both adjusted and unadjusted returns, 
 we will use adjusted ones everywhere.

In the case of a variable number of arguments, the check is a lot more 
complicated because excess arguments must be copied into a list and so
on; the entire process has been encapsulated in the millicode procedure
M\_VARARGS, and checking for a variable number of arguments involves
calling this procedure with an extra argument expressing the number of
fixed parameters (FIXED\_ARGS):

\begin{minipage}{\linewidth}
\begin{verbatim}
     jmpl   %MILLICODE + M_VARARGS, %o7
     mov    FIXED_ARGS, %ARGREG2
\end{verbatim}
\end{minipage}

\section{Stack Frame Creation}

For non-tail calls (and at some other times), the compiler will save
the live registers in a stack frame before the arguments to the call
are evaluated and the callee is invoked. The code for saving the registers
involves
allocating the stack frame (cheking for overflow in the process), and
then storing the necessary registers and a return address in the new
frame. The code is straightforward, if somewhat wordy; assume that
A\_FSIZE is the adjusted (up to a doubleword boundary) size of the
frame being allocated (in bytes), and FSIZE is the real size of the
frame:

\begin{minipage}{\linewidth}
\begin{verbatim}
    ld      [ %GLOBALS+SP_LIMIT_OFFSET ], %TMP0 ! Get stack limit
    cmp     %STKP, %TMP0                        ! check stack pointer
    bgt,a   L1                                  ! Skip if no overflow
    sub     %STKP, A_FSIZE, %STKP               ! Allocate frame (delayed)
    jmpl    %MILLICODE + M_STKOFLOW, %o7
    nop
    sub     %STKP, A_FSIZE, %STKP               ! Alloc frame (after trap)
L1:
    st      %REG0, [ %STKP+8 ]                  ! save REG0
    st      %REG1, [ %STKP+12 ]                 ! save REG1
      :
    ld      [ %GLOBALS + REG8_OFFSET ], %TMP0   ! fetch memory-mapped reg
    st      %TMP0, [ %STKP+40 ]                 ! save REG8
      :
    call    .+8
    add     %o7, (L2-(.-4))-8, %o7              ! return to some L2
    st      %o7, [ %STKP+0 ]                    ! save retrun address
    mov     FSIZE, %TMP0                        ! real frame size
    st      %TMP0, [ %STKP+4 ]                  ! save frame size
\end{verbatim}
\end{minipage}

Some of this code can be tweaked by using doubleword loads and stores in
several cases, but this is in general what a ``save'' looks like.

\section{Procedure Call}

Calling a procedure is fairly straightforward. The caller must decrement
the software timer (and raise an exception if it has reached zero), then check
to see that the callee is indeed a procedure (and raise an exception if it
is not), and finally it must set up the argument count register and branch
to the code for the callee.

Notice that the code for a tail call and the code for a non-tail call are
the same, as a stack frame (with a return address) is created separately
from the invocation of the callee.

\begin{minipage}{\linewidth}
\begin{verbatim}
L0:
     subcc  %TIMER, 1, %TIMER
     bne,a  L1
     and    %RESULT, TAG_MASK, %TMP0
     jmpl   %MILLICODE + M_TIMER_EXCEPTION, %o7
     add    %o7, L0-(.-4)-8, %o7
L1:
     cmp    %RESULT, PROC_TAG
     be,a   L2
     ld     [ %RESULT + CODEVECTOR_OFFSET - PROC_TAG ], %TMP0
     jmpl   %MILLICODE + M_PROC_EXCEPTION
     add    %o7, L1-(.-4)-8, %o7
L2:
     mov    %RESULT, %REG0
     jmp    %TMP0 + 4 - BVEC_TAG
     mov    ARGCOUNT, %RESULT
\end{verbatim}
\end{minipage}

(The bit about moving \%RESULT into \%REG0 is due to the MacScheme assembly
language calling conventions.) A call without exceptions takes 10 clocks if
the load hits the cache.

\section{Returning From a Call}

Returning from a call is almost trivial. The return address is in the
return address slot of the topmost stack frame, and all we have to do
is to get it out of there and jump to the indicated address. The result
will already have been stored in the RESULT register. There is no need
to pop a stack frame; this is done by the caller if necessary.

Sometimes the stack cache will underflow. To handle this, there is a dummy
continuation at the bottom of the stack which contains a return address to
the stack underflow handler. Hence, a procedure will normally return to its
caller, but sometimes it will be thrown into the underflow handler. The code
in the handler will restore a stack frame from the heap and jump to the
return address in the new frame. Neither caller nor callee will notice that
this has happened.

We therefore get the following code for the return operation:

\begin{minipage}{\linewidth}
\begin{verbatim}
    ld      [ %STKP + 0 ], %TMP0
    jmp     %TMP0+8
    nop
\end{verbatim}
\end{minipage}

Occasionally the delay slot can be filled; it is also advantageous to
try to move an instruction between the load and the jump to remove some
of the latency due to the load.

\section{Calling Millicode From Scheme}

All millicode procedures are called indirectly through the global
array "millicode", a pointer to which is in the register \%MILLICODE.
The offsets into the table are defined in the file "millicode.s.h".

Millicode procedures take 0, 1, 2, or 3 arguments. If there are
arguments, they are passed in the registers \%RESULT, \%ARGREG2, and
\%ARGREG3, in left-to-right order. All results are returned in the
\%RESULT register.  If there are no results, as in the case of e.g.
"setcar()", then \%RESULT is set to fixnum(~0~). \%ARGREG2 and \%ARGREG3 are
preserved across millicode calls and may hence be used as rootable
temporaries. [This guarantee is here because the memory management
millicode upholds it; if other millicode cannot, then we'll remove the
guarantee. The compiler will have to be somewhat sophisticated,
anyway, to use the argument registers as temps.]

Every millicode procedure requires that the return address is set up
in register \%o7 when the procedure is entered, and that \%REG0 has a
pointer to the procedure that called the millicode procedure.  [It
would be unreasonable for the millicode to require the return address
to be stored in the continuation, because the continuation may not
exist.]

To call a millicode procedure, then, we have the following code
snippets. First, assuming we're returning to the instruction
following the call:

\begin{minipage}{\linewidth}
\begin{verbatim}
    jmpl    %MILLICODE + M_<something>, %o7
    nop
\end{verbatim}
\end{minipage}

The delay slot can usually be filled with the setting up of one of 
the millicode argument registers.
Second, assuming we're returning to some other place (within the same
code vector!), in this case "L2":

\begin{minipage}{\linewidth}
\begin{verbatim}
    jmpl    %MILLICODE + M_<something>, %o7
    add     %o7, (L2-(.-4))-8, %o7
\end{verbatim}
\end{minipage}

\end{document}
