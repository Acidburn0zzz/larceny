% -*- LaTeX -*-

\documentstyle[10pt]{article}
\newcommand{\reg}[1]{{\sc \%#1}}

\topmargin      -2.0cm
\oddsidemargin   0.0cm
\evensidemargin  0.0cm
\textwidth      6.5in
\textheight     9.5in
\parindent       0.0cm
\parskip         0.4cm

\title{Larceny Note \#14: \\
       Thoughts on future enhancments \\
       {\tenrm (working document)}}
\author{Lars Thomas Hansen}

\begin{document}
\maketitle

\begin{abstract}
We present some thoughts on possible future enhancments which will improve the
run-time performance of Scheme programs running under Larceny. Primarily,
this document is to be considered an exhaustive list of things we want to
consider doing to the run-time system and/or the compiler, not all of which
should necessarily be incorporated.
\end{abstract}

\section{Keeping the Return Address in a Register}

Currently the return address is always stored in the stack frame when it has
been calculated by a {\tt save} or {\tt setrtn} instruction. In order to make
calls to leaf routines cheaper, it would be advantageous to keep the return
address in a register; this pays off especially when the compiler knows how
to reuse stack frames. A {\tt save} instruction would save the current return
address in the current stack frame before creating a new frame, and calculate
the new return address into a register. More return addresses can then be
calculated and put into the register with no memory traffic later; the previous
return address is already saved.

A different scenario is possible: the old return address is saved in the new
stack frame after this has been created.

In order to make either of these options work out, a number of conditions
must hold.

\begin{itemize}
\item
There must always be a stack frame in which to save the return address.
This makes things a little more complicated, and the second scenario would
be better in this respect.

\item
The return instruction does not need to load the return address since it's
already in a register.

\item
The {\tt save} and {\tt setrtn} instructions change.

\item
The {\tt restore} or the {\tt pop} instruction will need to fetch the return
address from a frame and move it back into the register.

\item
Millicode which knows about calling conventions will have to change.
In particular, {\tt save\_scheme\_context} will have to save the return
address either in the stack (scenario 1) or in a global (scenario 2).
In the case of scenario 2, it may be difficult to keep the reentrancy
guarantees we have had so far when going from Scheme to C to Scheme again.

\item
On a stack underflow, the underflow handler must get the return address
from the restored frame and put it into the return register.

\item
On a stack overflow, we must be careful to save the return address
in a register before the overflow handler is detected if we use the first
approach outlined above.

\end{itemize}

\section{Merging Millicode and Globals Tables}

This is fairly straightforward; we just need to allocate space for the globals
in {\tt tables.s} and adjust all offsets in {\tt offsets.cfg}. Eventually,
all references to one of the registers should go away and be replaced by
references to the other, but in the mean time, one can alias the other.

This change may or may not be desirable; it may be that we wish to keep them
separate so that all roots can later move into a vector in the heap.

\section{Fast Function Calls to Global Procedures}

Global variables should have a value cell and a function cell. When we are
calling a global variable, we should fetch the function cell rather than
the procedure cell, and just do a jump, assuming that the function cell
contains a procedure or the address of an error handler.

This involves work on the compiler (it must generate a special kind of {\tt
global} instruction), and {\tt setglbl} must be changed so that if it stores
a non-procedure into a global, it sets the function cell to the error
handler, and if it stores a procedure, it updates the function cell. It's
unclear how much we win, but it's probably worth the bother if we have lots
of global accessor functions and so on, all of which take little time and
which are all called frequently.

\end{document}