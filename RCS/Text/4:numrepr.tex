% -*- TeX -*-
% $Id: numrepr.txt,v 1.1 91/12/06 14:59:03 lth Exp Locker: lth $

% -*- LaTeX -*-

\documentstyle[10pt]{article}
\newcommand{\reg}[1]{{\sc \%#1}}

\topmargin      -2.0cm
\oddsidemargin   0.0cm
\evensidemargin  0.0cm
\textwidth      6.5in
\textheight     9.5in
\parindent       0.0cm
\parskip         0.4cm


\title{Larceny Note \# 4: \\
       Numeric Representations}
\author{Lars Thomas Hansen}

\begin{document}
\maketitle

Fixnums are unboxed and kept in the high 30 bits of a word, with the
two low bits always 0:

\begin{verbatim}
        +------------------------------+--+
        |       fixnum                 |00|
        +------------------------------+--+
\end{verbatim}

Bignums are bytevector-like with the sign in the first two bytes
(\#x0000 for 0 or positive, \#xFFFF for negative), followed by a digit
count (two bytes) and then base-$2^32$ digits in the next words.
with the least significant word first. Each word is stored big-endian,
and can be interpreted as two 16-bit digits, also in big-endian
fashion.

\begin{verbatim}
        +------------------------+--------+
        |       length           | header |
        +------------------------+--------+
        | sign          |   digitcount    |
        +---------------------------------+
        |              lsd                |
        +---------------------------------+
        ...
\end{verbatim}

Ratnums are vector-like, with the first word of the vector being the
numerator as a scheme object (fixnum or bignum), and the second word
being the denominator (greater than 1) (ditto):

\begin{verbatim}
        +------------------------+--------+
        |       vectorlength     | header |
        +------------------------+--------+
        |       numerator                 |
        +---------------------------------+
        |       denominator               |
        +---------------------------------+
\end{verbatim}

Rectnums look like ratnums, except that the first word is the real
part (an integer or ratnum) and the second word is the imaginary part
(ditto): (both exact reals; imag part is nonzero)

\begin{verbatim}
        +------------------------+--------+
        |       vectorlength     | header |
        +------------------------+--------+
        |       real-part                 |
        +---------------------------------+
        |       imag-part                 |
        +---------------------------------+
\end{verbatim}

Flonums (IEEE double) are bytevector-like. The first word is unused.
The two next words contain the double:

\begin{verbatim}
        +------------------------+--------+
        |      length            | header |
        +------------------------+--------+
        |      unused                     |
        +---------------------------------+
        |      IEEE double precision      |
        |                                 |
        +---------------------------------+
\end{verbatim}

Compnums (two IEEE doubles) are bytevector-like. The first word is
unused.  The two next words contain the real part. The two last words
contain the imaginary part:

\begin{verbatim}
        +------------------------+--------+
        |      length            | hdrtag |
        +------------------------+--------+
        |      unused                     |
        +---------------------------------+
        |      IEEE double precision      |
        |      (real part)                |
        +---------------------------------+
        |      IEEE double precision      |
        |      (imaginary part)           |
        +---------------------------------+
\end{verbatim}

The rationale of the unused part of the flonums and compnums is that
since everything in the is aligned on an 8-byte boundary, a word will
be wasted anyway, and when we insert that word in the structure rather
than having it trailing the structure, we can double-load the flonums
if desired.

\end{document}
