% -*- TeX -*-
% $Id: numrepr.tex,v 1.2 92/01/24 14:04:47 lth Exp Locker: lth $

% -*- LaTeX -*-

\documentstyle[10pt]{article}
\newcommand{\reg}[1]{{\sc \%#1}}

\topmargin      -2.0cm
\oddsidemargin   0.0cm
\evensidemargin  0.0cm
\textwidth      6.5in
\textheight     9.5in
\parindent       0.0cm
\parskip         0.4cm


\title{Larceny Note \#4: \\
       Number Representations \\
       {\tenrm (definitive!)}}
\author{Lars Thomas Hansen}

\begin{document}
\maketitle

Fixnums (small exact integers) are unboxed and kept in the high 30
bits of a word, with the two low bits always 0 (figure 1).

\begin{minipage}{\linewidth}
\begin{verbatim}
                          +------------------------------+--+
                          |       fixnum                 |00|
                          +------------------------------+--+
\end{verbatim}
\centerline{Figure 1: Fixnum}
\end{minipage}

Bignums (large exact integers) are bytevector-like with the sign in
the first two bytes (0 for positive, 1 for negative), followed by a
digit count (two bytes) and then base-$2^{32}$ digits in the next
words, with the least significant word first; each word is stored
big-endian (figure 2).  While bignums cannot be 0 in user code, system
code sometimes creates bignums of value 0 in an internal calculation.
A bignum with value 0 is distinguished by a digitcount of 0; the sign
is immaterial.

\begin{minipage}{\linewidth}
\begin{verbatim}
                          +------------------------+--------+
                          |       length           | hdrtag |
                          +------------------------+--------+
                          | sign          |   digitcount    |
                          +---------------------------------+
                          |              lsd                |
                          +---------------------------------+
                          ...
\end{verbatim}
\centerline{Figure 2: Bignum with 32-bit digits}
\end{minipage}

The rationale for keeping a digit count which is different from the vector
length is that while (in particular) the multiplication routine must create
a vector whose length is the sum of the digit counts, some of the leading
digits may be 0, and hence we can have a lower digit count without having
to reallocate memory or use a temporary buffer.

Bignums can also be considered with a different logical layout: Each
32-bit digit can be interpreted as two 16-bit digits, also in
big-endian fashion within the word; interpreted this way, the bignum
gets a funny access pattern (figure 3).  The digit count is still the
number of 32-bit digits used; see above discussion for sign and
bignums of value 0.

\begin{minipage}{\linewidth}
\begin{verbatim}
                          +------------------------+--------+
                          |       length           | hdrtag |
                          +------------------------+--------+
                          | sign          |   digitcount    |
                          +---------------------------------+
                          |    nlsd       |      lsd        |
                          +---------------------------------+
                          ...
\end{verbatim}
\centerline{Figure 3: Bignum with 16-bit digits}
\end{minipage}

Ratnums (exact rationals), shown in figure 4, are vector-like, with
the first word of the vector being the numerator as a Scheme object
(fixnum or bignum), and the second word being the denominator (greater
than 1), also a Scheme object.

\begin{minipage}{\linewidth}
\begin{verbatim}
                          +------------------------+--------+
                          |       vectorlength     | hdrtag |
                          +------------------------+--------+
                          |       numerator                 |
                          +---------------------------------+
                          |       denominator               |
                          +---------------------------------+
\end{verbatim}
\centerline{Figure 4: Ratnum}
\end{minipage}

Rectnums (exact complexes), shown in figure 5, look like ratnums,
except that the first word is the real part (an integer or ratnum) and
the second word is the imaginary part (ditto). Both parts are exact
reals, and the imaginary part is nonzero.

\begin{minipage}{\linewidth}
\begin{verbatim}
                          +------------------------+--------+
                          |       vectorlength     | hdrtag |
                          +------------------------+--------+
                          |       real-part                 |
                          +---------------------------------+
                          |       imag-part                 |
                          +---------------------------------+
\end{verbatim}
\centerline{Figure 5: Rectnum}
\end{minipage}

Flonums (inexact reals) are bytevector-like. The first word is unused,
and the two next words contain the double in IEEE double precision
format. The rationale for the unused word is this: since everything in
the system is aligned on an 8-byte boundary, one word will be wasted
anyway. By putting it before the number rather than after, the number
becomes 8-byte aligned, and we can use a ``load double'' instruction
to load it. (Figure 6.)

\begin{minipage}{\linewidth}
\begin{verbatim}
                          +------------------------+--------+
                          |      length            | hdrtag |
                          +------------------------+--------+
                          |      unused                     |
                          +---------------------------------+
                          |      IEEE double precision      |
                          |                                 |
                          +---------------------------------+
\end{verbatim}
\centerline{Figure 6: Flonum}
\end{minipage}

Compnums (inexact complexes) are bytevector-like. The first word is
unused (see the description of the flonum for a rationale).  The two
next words contain the real part. The two last words contain the
imaginary part. (Figure 7.) Both parts are IEEE double precision
numbers.

\begin{minipage}{\linewidth}
\begin{verbatim}
                          +------------------------+--------+
                          |      length            | hdrtag |
                          +------------------------+--------+
                          |      unused                     |
                          +---------------------------------+
                          |      IEEE double precision      |
                          |      (real part)                |
                          +---------------------------------+
                          |      IEEE double precision      |
                          |      (imaginary part)           |
                          +---------------------------------+
\end{verbatim}
\centerline{Figure 7: Compnum}
\end{minipage}

\end{document}
